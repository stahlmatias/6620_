\chapter{Arquitectura de programación}
El término arquitectura del conjunto de instrucciones (instruction set architecture) (ISA) se refiere al conjunto de instrucciones visibles al programador. Funciona como la frontera entre el software y el hardware.


\section{Modelo de pila}
Los operandos en una arquitectura de stack están implícitamente en el tope del stack y el hardware debe evaluar la expresión en un solo orden y cargar un operando múltiples veces.


\section{Modelo de acumulador}
En una arquitectura de acumulador un operando está implícito en el acumulador.


\section{Modelo de registros de propósito general}
En la arquitectura de registros de propósito general se tienen únicamente operandos explícitos, ya sea que estén ubicados en registros o en memoria.

Los operandos explícitos podrían ser accedidos directamente de memoria o necesitar ser cargados de memoria a un registro temporal dependiendo de la arquitectura y de la instrucción específica.


\section{Modelo de carga y almacenamiento}

En este tipo de arquitectura, la memoria solo puede ser accedida a través de instrucciones de carga y almacenamiento (load/store). 

\subsection{Conjunto de instrucciones RISC}

Las arquitecturas RISC se caracterizan por tener ciertas propiedades que simplifican su implementación.

\begin{itemize}
\item Todas las operaciones de datos aplican a datos en registros y típicamente modifican todo el registro (32 o 64 bits por registro).
\item Las únicas operaciones que afectan a la memoria son de carga y almacenamiento (load y stores) que mueven datos de la memoria a un registro o viceversa. Estas operaciones pueden mover datos menores que la capacidad de un registro (por ejemplo 16 bits).
\item El número de formatos de instrucciones son pocos con todas las instrucciones del mismo tamaño.
\end{itemize}

Como otras arquitecturas RISC, el conjunto de instrucciones de MIPS tiene 32 registros, aunque el registro 0 siempre tiene el valor 0. 

Hay tres tipos de instrucciones:

\begin{enumerate}
\item Instrucciones ALU. Estas instrucciones toman dos registros o bien un registro y un inmediato de signo extendido, realiza la operación y almacena el resultado en otro registro.

\item Instrucciones Load / Store. Estas instrucciones toman un registro base, y un inmediato como offset. La suma resultante del contenido del registro y el offset da la dirección efectiva y es usada para direccionar la memoria. En el caso de un store, el segundo registro opera como fuente del dato que se almacena en memoria.

\item Branches  y  jumps. Branches son transferencias condicionales del flujo de control. La forma de especificar la condición del branch, MIPS compara entre un par de registros o entre un registro y cero.

El destino del branch se obtiene sumando un offset con extensión de signo (16 bits) al valor actual del PC.
\end{enumerate}

\subsection{Modos de direccionamiento}
\subsection{Formato de instrucciones}
